\documentclass[12pt]{article}
\usepackage[margin=1in]{geometry}
\usepackage{amsmath, amssymb, mathtools}
\usepackage{tikz}
\usepackage{lmodern}
\usepackage{hyperref}
\usepackage{caption}
\usepackage{float}
\usepackage{parskip}
\usepackage{tabularx}
\usepackage[utf8]{inputenc}
\usepackage{tgpagella}
\usepackage[T1]{fontenc}
\usepackage{array}
\usepackage{booktabs}
\usepackage{tcolorbox}
\usepackage{enumitem}

\newcommand{\separator}{\noindent\rule{\linewidth}{1pt}}

\title{Think Again: How to Understand Arguments}
\author{Zubin}
\date{\today}

\begin{document}

\maketitle

\tableofcontents

\newpage

\separator
%-----------------------------------------------------------------

\section{Introduction}
\textbf{Course:} Think Again: How to Reason and Argue \\
\textbf{Instructors:} Walter Sinnott-Armstrong (Duke University) and Ram Neta (UNC Chapel Hill).

\subsection{Why Arguments Matter}
Arguments are important for two main reasons:
\begin{enumerate}
    \item \textbf{To have reasons for beliefs and actions:} Arguments express reasons. Understanding them helps avoid arbitrary beliefs and provides justification for thoughts and behaviors.
    \item \textbf{To avoid mistakes:} It helps in spotting bad arguments from others (e.g., salespeople, lawyers, evangelists) and prevents being misled. It also aids in making well-thought-out personal decisions.
\end{enumerate}

\subsection{Course Structure}
The course is divided into four parts:
\begin{description}
    \item[Part 1: Analyzing Arguments] \hfill \\
          Distinguishing arguments from non-arguments, identifying argument parts, using standard form, and finding suppressed premises.

    \item[Part 2: Deductive Arguments] \hfill \\
          Focuses on valid, formal structures including propositional logic and categorical logic.

    \item[Part 3: Inductive Arguments] \hfill \\
          Includes statistical generalizations, inference to the best explanation, arguments from analogy, causal reasoning, probability, and decision making.

    \item[Part 4: Fallacies] \hfill \\
          Covers common mistakes like vagueness, ambiguity, irrelevance (ad hominem, appeals to ignorance), and begging the question.
\end{description}

\subsection{Logistics}
\begin{itemize}
    \item \textbf{Communication:} Use discussion forums for questions and interaction. Do not email instructors individually.
\end{itemize}

\subsection{Recommended Reading}
For those who want more detail or are having trouble understanding:
\begin{itemize}
    \item \textbf{Textbook:} \textit{Understanding Arguments}
    \item \textbf{Authors:} Walter Sinnott-Armstrong and Robert Fogelin
\end{itemize}

\separator

\section{What Is an Argument?}
To understand arguments, we must first distinguish them from what they are not.

\subsection{What Arguments Are Not}
Arguments are distinct from:
\begin{itemize}
    \item \textbf{Physical Fights:} Hitting someone is not arguing.
    \item \textbf{Verbal Fights:} Yelling is not arguing.
    \item \textbf{Abuse:} Name-calling (e.g., ``stupid git'') is not arguing.
    \item \textbf{Complaining:} Expressing emotion is not arguing.
    \item \textbf{Contradiction:} Simply denying a statement is not arguing.
\end{itemize}

\subsection{Defining an Argument}
Monty Python defines an argument as ``a connected series of statements to establish a proposition.'' However, this is too narrow because arguments can also explain propositions we already accept (e.g., mathematical proofs).

\textbf{Broader Definition:} An argument is a connected series of sentences, statements, or propositions, where some are premises and one is the conclusion, and the premises are intended to provide some kind of reason for the conclusion.

This definition highlights:
\begin{enumerate}
    \item \textbf{Parts:} Premises and Conclusion.
    \item \textbf{Medium:} Language.
    \item \textbf{Purpose:} To give a reason for the conclusion.
    \item \textbf{Flexibility:} Covers different kinds of reasons (justification and explanation).
\end{enumerate}


\separator

\section{Quiz – What is Argument}

\subsection{1. Arguments are verbal fights.}
\begin{itemize}[label={}]
    \item $\square$ True
    \item $\boxtimes$ False
\end{itemize}
\textbf{Correct.} \\
\textbf{Answer: False.} People in fights are trying to hurt each other, but people who give arguments are often trying to help each other.

\subsection{2. Every argument includes a conclusion.}
\begin{itemize}[label={}]
    \item $\boxtimes$ True
    \item $\square$ False
\end{itemize}
\textbf{Correct.} \\
\textbf{Answer: True.} Arguments are defined so that they must always have a conclusion.

\subsection{3. All arguments are made up of (or expressed in) language.}
\begin{itemize}[label={}]
    \item $\boxtimes$ True
    \item $\square$ False
\end{itemize}
\textbf{Correct.} \\
\textbf{Answer: True.} Premises are sentences, statements, or propositions. Sentences and statements are made up of language, and propositions are expressed by language. Notice that no particular language is required, so it would be false to say that all arguments are in English.

\subsection{4. Every argument is intended to establish a conclusion that the audience did not believe before.}
\begin{itemize}[label={}]
    \item $\square$ True
    \item $\boxtimes$ False
\end{itemize}
\textbf{Correct.} \\
\textbf{Answer: False.} Sometimes the conclusion is already both believed and established as true, and the point of the argument is only to explain \emph{why} it is true. Hence, Monty Python is not always right.

\subsection{5. Every argument succeeds in giving good reasons for its conclusion.}
\begin{itemize}[label={}]
    \item $\square$ True
    \item $\boxtimes$ False
\end{itemize}
\textbf{Correct.} \\
\textbf{Answer: False.} Although people who give arguments always intend to give some kind of reason, they often fail to fulfill that intention. Bad arguments fail to give good reasons.

\subsection{6. Megafauna: n. very large animals.}
\begin{itemize}[label={}]
    \item $\square$ Yes, this is an argument.
    \item $\boxtimes$ No, this is not an argument.
\end{itemize}
\textbf{Correct.} \\
\textbf{Answer: No.} The defined term is not a full sentence, statement, or proposition, so it cannot be a conclusion. The definition is not a reason for the term that is defined.

\subsection{7. Reptiles include turtles, alligators, crocodiles, snakes, lizards, and the tuatara.}
\begin{itemize}[label={}]
    \item $\square$ Yes, this is an argument.
    \item $\boxtimes$ No, this is not an argument.
\end{itemize}
\textbf{Correct.} \\
\textbf{Answer: No.} One word in a list does not give a reason for the other words. One could argue, ``This is a turtle, so it is a reptile,'' but the list by itself does not explicitly state that argument or any argument.

\subsection{8. World War II occurred after World War I occurred.}
\begin{itemize}[label={}]
    \item $\square$ Yes, this is an argument.
    \item $\boxtimes$ No, this is not an argument.
\end{itemize}
\textbf{Correct.} \\
\textbf{Answer: No.} This sentence is about historical or chronological order rather than rational order. It does not explicitly claim that World War I gives a reason for World War II.

\subsection{9. World War II occurred because World War I occurred.}
\begin{itemize}[label={}]
    \item $\boxtimes$ Yes, this is an argument.
    \item $\square$ No, this is not an argument.
\end{itemize}
\textbf{Correct.} \\
\textbf{Answer: Yes.} The word ``because'' makes this sentence claim that World War I gives a reason why World War II occurred. The premise is ``World War I occurred,'' and the conclusion is ``World War II occurred.''

\subsection{10. The sides of this right triangle are 1 meter long, so its hypotenuse is 2 meters long.}
\begin{itemize}[label={}]
    \item $\boxtimes$ Yes, this is an argument.
    \item $\square$ No, this is not an argument.
\end{itemize}
\textbf{Correct.} \\
\textbf{Answer: Yes.} The word ``so'' indicates that the first sentence is supposed to be a reason for the second sentence. This argument is very bad, since the hypotenuse must be $\sqrt{2}$ meters long instead of 2 meters long. But bad arguments are still arguments.




\separator

\section{What Are Arguments Used For?}
To understand arguments, we need to understand their purposes. The purpose is crucial in determining what an object is (e.g., distinguishing a screwdriver from a spatula based on intent).

\subsection{Persuasion}
One purpose of an argument is to convince someone to do or believe something they wouldn't otherwise.
\begin{itemize}
    \item \textbf{Goal:} To change mental states (beliefs) or behaviors. It aims to bring about an effect in the world.
    \item \textbf{Example:} A used car salesman listing features (cool look, fast speed) to convince you to buy a Mustang.
    \item \textbf{Criteria:} Success is measured by whether the person is convinced, regardless of whether the reasons are good or bad.
\end{itemize}

\subsection{Justification}
Justification involves giving a reason for a belief or action, not necessarily to convince or persuade.
\begin{itemize}
    \item \textbf{Goal:} To provide good reasons for a belief or decision.
    \item \textbf{Example:} A friend discussing the pros and cons of a car to help you make your own decision.
    \item \textbf{Criteria:} Success depends on whether the reasons provided are \textit{good} reasons.
\end{itemize}

\textbf{Key Distinction:}
\begin{itemize}
    \item \textbf{Persuasion:} Focuses on the \textit{effect} on the audience. Bad arguments can be persuasive.
    \item \textbf{Justification:} Focuses on the \textit{rational support} for the conclusion. Requires good arguments.
\end{itemize}

\separator

\section{Strong Arguments Don't Always Persuade Everyone}

\begin{tcolorbox}
    \textbf{Strong arguments don't always persuade everyone} \\
    \textit{by Jessica Hyde from the United Kingdom}
    \vspace{1em}

    It's not enough for an argument to be strong, valid and sound to be persuasive. You can have an argument for which every premise is genuinely true, and where every conceivable flaw in the argument is negated and still, not have it be persuasive.

    There will almost always be someone who either misunderstands the argument, or blindly believes the opposite of a premise, in face of facts. Human beings aren't always logical and don't always believe scientifically proven cause and effect. Religious and cultural beliefs can be too hard to overcome. So even the best arguments can have disbelievers.
\end{tcolorbox}

\begin{tcolorbox}
    \textbf{The Benchmark of Success: Understanding} \\
    \textit{by Judith}
    \vspace{1em}

    I think Jessica has opened a very interesting discussion with her argument. Thank you Jessica, I appreciate that, we do too. When I'm learning, is the purpose of an argument is to state with clarity, and some degree of certainty, an opinion or point of view; a valid, strong and sound argument in it of itself may never persuade or convert anyone to adopt a different way of thinking. So what. What a strong argument does is communicate clearly what one thinks and why they think it.

    So I guess the benchmark of success for many arguments is not complete persuasion, but is how clearly one is understood. If someone's intent is to blindly refute everything, that's not an intellectually honest engagement. I've found that in construction better, more thoughtful arguments people may not agree with me, but they're far more considerate of what I have to say. And by using much of what we're learning, I'm listening much more intently to other views. Yes, Jessica, many things do defy logic. We just keep trying to do our best.
\end{tcolorbox}

\subsection{The Limits of Persuasion}
As student Jessica Hyde argued, a strong, valid, and sound argument may not be persuasive.
\begin{itemize}
    \item \textbf{Obstacles:} People may misunderstand the argument, blindly believe the opposite, or be influenced by religious/cultural beliefs that override logic.
    \item \textbf{Implication:} You can succeed in \textit{justifying} your conclusion (giving good reasons) while failing to \textit{persuade} the audience.
\end{itemize}

\subsection{Understanding as a Goal}
Student Judith highlighted a third purpose of argument: \textbf{Understanding}.
\begin{itemize}
    \item \textbf{Goal:} To state clearly what one thinks and why, even if the audience is not converted.
    \item \textbf{Benefit:} It fosters respect and consideration. If we understand why we disagree, we can get along better and seek compromise (unlike the yelling often seen in politics).
\end{itemize}

\subsection{Conclusion on Purposes}
If your goal is to persuade everyone, you will be frustrated. Instead, reasonable goals include:
\begin{itemize}
    \item \textbf{Justification:} Giving good reasons.
    \item \textbf{Understanding:} Helping others understand your position.
\end{itemize}


\separator

\section{Quiz – Strong Arguments Don't Always Persuade Everyone}

\subsection{1. If your argument does not persuade your audience, it is no good.}
\begin{itemize}[label={}]
    \item $\square$ True
    \item $\boxtimes$ False
\end{itemize}
\textbf{Correct.} \\
\textbf{Answer: False.} An argument can still be good for the purposes of justification and explanation even if nobody is persuaded by it. It can also help one’s audience understand one’s position, even if they are not persuaded, as Judith said in the lecture.

\subsection{2. An argument that does not give any good reason to believe its conclusion can still persuade someone to believe its conclusion.}
\begin{itemize}[label={}]
    \item $\boxtimes$ True
    \item $\square$ False
\end{itemize}
\textbf{Correct.} \\
\textbf{Answer: True.} People can get fooled by bad reasons.

\subsection{3. Sometimes people cannot be persuaded by very strong arguments because they refuse to give up their beliefs.}
\begin{itemize}[label={}]
    \item $\boxtimes$ True
    \item $\square$ False
\end{itemize}
\textbf{Correct.} \\
\textbf{Answer: True.} Refusing to give up a strongly held belief can prevent someone from being persuaded, as Jessica says in the lecture. Sometimes those strongly held beliefs are false or unjustified, but sometimes they are true and justified. In either case, these people are not persuaded, because persuasion requires a change in belief.

\subsection{4. When people use arguments, they always intend to have some effect on other people.}
\begin{itemize}[label={}]
    \item $\square$ True
    \item $\boxtimes$ False
\end{itemize}
\textbf{Correct.} \\
\textbf{Answer: False.} Sometimes we formulate arguments in private in order to figure out what to believe ourselves without telling anyone else.

\subsection{5. Can any argument persuade every person in the world?}
\begin{itemize}[label={}]
    \item $\square$ Yes
    \item $\boxtimes$ No
\end{itemize}
\textbf{Correct.} \\
\textbf{Answer: No.} Every argument has to be formulated in some language, but there will always be someone in the world who does not speak that language or understand that argument, so they cannot be persuaded by it. Babies are also people who cannot be persuaded by arguments, at least if they do not yet speak any language.

\separator

\section{What Else are Arguments Used For? Explanation}
Persuasion and justification are not the only purposes of arguments. Arguments are also used to \textbf{explain} things.

\subsection{The Goal of Explanation}
Unlike persuasion or justification, explanation assumes the conclusion is already true (e.g., asking ``Why did Duke win?'' implies they did win).
\begin{itemize}
    \item \textbf{Goal:} To increase \textbf{understanding} of why something happened, fitting the phenomenon into a general pattern to remove bewilderment.
    \item \textbf{Distinction:} You don't try to convince someone \textit{that} it happened (persuasion), but \textit{why} it happened.
\end{itemize}

\subsection{Aristotle's Four Types of Explanation}
Aristotle identified four types of causes or explanations. We can apply all four to a single event, such as a train whistle:
\begin{enumerate}
    \item \textbf{Causal (Efficient):} The event that brought it about (e.g., The conductor pulled the lever).
    \item \textbf{Teleological (Purpose):} The goal or purpose (e.g., To warn cars at an intersection).
    \item \textbf{Formal:} Based on shape or form (e.g., The whistle's shape creates the vibration).
    \item \textbf{Material:} Based on the material composition (e.g., The air density creates the sound).
\end{enumerate}

\textbf{Another Example (Joe jumping out of an airplane):}
\begin{itemize}
    \item \textbf{Causal:} He jumped (the event that caused the fall).
    \item \textbf{Teleological:} To get excitement (the purpose of jumping).
    \item \textbf{Formal:} His aerodynamic shape (explains why he fell fast).
    \item \textbf{Material:} His density being greater than air (explains why he fell).
\end{itemize}

\subsection{Explanation as an Argument}
Explanations can be narratives, but these don't yield general principles. In this course, we focus on explanations as arguments:
\begin{itemize}
    \item \textbf{Structure:} A general principle + Particular facts $\rightarrow$ Conclusion (The event explained).
    \item \textbf{Example (Helium Balloon):}
          \begin{itemize}
              \item \textit{Principle:} Objects less dense than the medium rise.
              \item \textit{Fact:} This balloon is less dense than air.
              \item \textit{Conclusion:} Therefore, the balloon rises.
          \end{itemize}
\end{itemize}

\subsection{Important Distinctions}
\begin{itemize}
    \item \textbf{Prediction vs. Explanation:} \textbf{Bode's Law} predicted planetary distances accurately (for known planets) but did not explain \textit{why} they were positioned that way. It offered prediction without explanation.
    \item \textbf{Explanation vs. Prediction/Justification:} An HIV-positive mother explains why a child is HIV-positive. However, this doesn't \textit{predict} it (less than 50\% chance) nor \textit{justify} the belief (a blood test is needed for that).
    \item \textbf{Generalization:} Explanation is also distinct from simple generalization.
\end{itemize}

\end{document}
