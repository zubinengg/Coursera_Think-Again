\documentclass[12pt]{article}
\usepackage[margin=1in]{geometry}
\usepackage{amsmath, amssymb, mathtools}
\usepackage{tikz}
\usepackage{lmodern}
\usepackage{hyperref}
\usepackage{caption}
\usepackage{float}
\usepackage{parskip}
\usepackage{tabularx}
\usepackage[utf8]{inputenc}
\usepackage{tgpagella}
\usepackage[T1]{fontenc}
\usepackage{array}
\usepackage{booktabs}
\usepackage{tcolorbox}
\usepackage{enumitem}

\newcommand{\separator}{\noindent\rule{\linewidth}{1pt}}

\title{Think Again: How to Understand Arguments}
\author{Zubin}
\date{\today}

\begin{document}

\maketitle

\tableofcontents

\newpage

\separator
%-----------------------------------------------------------------

\section{Introduction}
\textbf{Course:} Think Again: How to Reason and Argue \\
\textbf{Instructors:} Walter Sinnott-Armstrong (Duke University) and Ram Neta (UNC Chapel Hill).

\subsection{Why Arguments Matter}
Arguments are important for two main reasons:
\begin{enumerate}
    \item \textbf{To have reasons for beliefs and actions:} Arguments express reasons. Understanding them helps avoid arbitrary beliefs and provides justification for thoughts and behaviors.
    \item \textbf{To avoid mistakes:} It helps in spotting bad arguments from others (e.g., salespeople, lawyers, evangelists) and prevents being misled. It also aids in making well-thought-out personal decisions.
\end{enumerate}

\subsection{Course Structure}
The course is divided into four parts:
\begin{description}
    \item[Part 1: Analyzing Arguments] \hfill \\
          Distinguishing arguments from non-arguments, identifying argument parts, using standard form, and finding suppressed premises.

    \item[Part 2: Deductive Arguments] \hfill \\
          Focuses on valid, formal structures including propositional logic and categorical logic.

    \item[Part 3: Inductive Arguments] \hfill \\
          Includes statistical generalizations, inference to the best explanation, arguments from analogy, causal reasoning, probability, and decision making.

    \item[Part 4: Fallacies] \hfill \\
          Covers common mistakes like vagueness, ambiguity, irrelevance (ad hominem, appeals to ignorance), and begging the question.
\end{description}

\subsection{Logistics}
\begin{itemize}
    \item \textbf{Communication:} Use discussion forums for questions and interaction. Do not email instructors individually.
\end{itemize}

\subsection{Recommended Reading}
For those who want more detail or are having trouble understanding:
\begin{itemize}
    \item \textbf{Textbook:} \textit{Understanding Arguments}
    \item \textbf{Authors:} Walter Sinnott-Armstrong and Robert Fogelin
\end{itemize}

\separator

\section{What Is an Argument?}
To understand arguments, we must first distinguish them from what they are not.

\subsection{What Arguments Are Not}
Arguments are distinct from:
\begin{itemize}
    \item \textbf{Physical Fights:} Hitting someone is not arguing.
    \item \textbf{Verbal Fights:} Yelling is not arguing.
    \item \textbf{Abuse:} Name-calling (e.g., ``stupid git'') is not arguing.
    \item \textbf{Complaining:} Expressing emotion is not arguing.
    \item \textbf{Contradiction:} Simply denying a statement is not arguing.
\end{itemize}

\subsection{Defining an Argument}
Monty Python defines an argument as ``a connected series of statements to establish a proposition.'' However, this is too narrow because arguments can also explain propositions we already accept (e.g., mathematical proofs).

\textbf{Broader Definition:} An argument is a connected series of sentences, statements, or propositions, where some are premises and one is the conclusion, and the premises are intended to provide some kind of reason for the conclusion.

This definition highlights:
\begin{enumerate}
    \item \textbf{Parts:} Premises and Conclusion.
    \item \textbf{Medium:} Language.
    \item \textbf{Purpose:} To give a reason for the conclusion.
    \item \textbf{Flexibility:} Covers different kinds of reasons (justification and explanation).
\end{enumerate}
\section{Quiz – What is Argument}

\subsection{1. Arguments are verbal fights.}
\begin{itemize}[label={}]
    \item $\square$ True
    \item $\boxtimes$ False
\end{itemize}
\textbf{Correct.} \\
\textbf{Answer: False.} People in fights are trying to hurt each other, but people who give arguments are often trying to help each other.

\subsection{2. Every argument includes a conclusion.}
\begin{itemize}[label={}]
    \item $\boxtimes$ True
    \item $\square$ False
\end{itemize}
\textbf{Correct.} \\
\textbf{Answer: True.} Arguments are defined so that they must always have a conclusion.

\subsection{3. All arguments are made up of (or expressed in) language.}
\begin{itemize}[label={}]
    \item $\boxtimes$ True
    \item $\square$ False
\end{itemize}
\textbf{Correct.} \\
\textbf{Answer: True.} Premises are sentences, statements, or propositions. Sentences and statements are made up of language, and propositions are expressed by language. Notice that no particular language is required, so it would be false to say that all arguments are in English.

\subsection{4. Every argument is intended to establish a conclusion that the audience did not believe before.}
\begin{itemize}[label={}]
    \item $\square$ True
    \item $\boxtimes$ False
\end{itemize}
\textbf{Correct.} \\
\textbf{Answer: False.} Sometimes the conclusion is already both believed and established as true, and the point of the argument is only to explain \emph{why} it is true. Hence, Monty Python is not always right.

\subsection{5. Every argument succeeds in giving good reasons for its conclusion.}
\begin{itemize}[label={}]
    \item $\square$ True
    \item $\boxtimes$ False
\end{itemize}
\textbf{Correct.} \\
\textbf{Answer: False.} Although people who give arguments always intend to give some kind of reason, they often fail to fulfill that intention. Bad arguments fail to give good reasons.

\subsection{6. Megafauna: n. very large animals.}
\begin{itemize}[label={}]
    \item $\square$ Yes, this is an argument.
    \item $\boxtimes$ No, this is not an argument.
\end{itemize}
\textbf{Correct.} \\
\textbf{Answer: No.} The defined term is not a full sentence, statement, or proposition, so it cannot be a conclusion. The definition is not a reason for the term that is defined.

\subsection{7. Reptiles include turtles, alligators, crocodiles, snakes, lizards, and the tuatara.}
\begin{itemize}[label={}]
    \item $\square$ Yes, this is an argument.
    \item $\boxtimes$ No, this is not an argument.
\end{itemize}
\textbf{Correct.} \\
\textbf{Answer: No.} One word in a list does not give a reason for the other words. One could argue, ``This is a turtle, so it is a reptile,'' but the list by itself does not explicitly state that argument or any argument.

\subsection{8. World War II occurred after World War I occurred.}
\begin{itemize}[label={}]
    \item $\square$ Yes, this is an argument.
    \item $\boxtimes$ No, this is not an argument.
\end{itemize}
\textbf{Correct.} \\
\textbf{Answer: No.} This sentence is about historical or chronological order rather than rational order. It does not explicitly claim that World War I gives a reason for World War II.

\subsection{9. World War II occurred because World War I occurred.}
\begin{itemize}[label={}]
    \item $\boxtimes$ Yes, this is an argument.
    \item $\square$ No, this is not an argument.
\end{itemize}
\textbf{Correct.} \\
\textbf{Answer: Yes.} The word ``because'' makes this sentence claim that World War I gives a reason why World War II occurred. The premise is ``World War I occurred,'' and the conclusion is ``World War II occurred.''

\subsection{10. The sides of this right triangle are 1 meter long, so its hypotenuse is 2 meters long.}
\begin{itemize}[label={}]
    \item $\boxtimes$ Yes, this is an argument.
    \item $\square$ No, this is not an argument.
\end{itemize}
\textbf{Correct.} \\
\textbf{Answer: Yes.} The word ``so'' indicates that the first sentence is supposed to be a reason for the second sentence. This argument is very bad, since the hypotenuse must be $\sqrt{2}$ meters long instead of 2 meters long. But bad arguments are still arguments.



\end{document}