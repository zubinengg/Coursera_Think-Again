\documentclass[12pt]{article}
\usepackage[margin=1in]{geometry}
\usepackage{amsmath, amssymb, mathtools}
\usepackage{tikz}
\usepackage{lmodern}
\usepackage{hyperref}
\usepackage{caption}
\usepackage{float}
\usepackage{parskip}
\usepackage{tabularx}
\usepackage[utf8]{inputenc}
\usepackage{tgpagella}
\usepackage[T1]{fontenc}
\usepackage{array}
\usepackage{booktabs}
\usepackage{tcolorbox}
\usepackage{enumitem}

\title{Recommended Books for In-Depth Knowledge}
\author{Zubin}
\date{\today}

\begin{document}

\maketitle

\tableofcontents

\newpage

\noindent\rule{\linewidth}{1pt}
%-----------------------------------------------------------------

\section{Introduction}
\textbf{Course:} Think Again: How to Reason and Argue \\
\textbf{Instructors:} Walter Sinnott-Armstrong (Duke University) and Ram Neta (UNC Chapel Hill).

\subsection{Why Arguments Matter}
Arguments are important for two main reasons:
\begin{enumerate}
    \item \textbf{To have reasons for beliefs and actions:} Arguments express reasons. Understanding them helps avoid arbitrary beliefs and provides justification for thoughts and behaviors.
    \item \textbf{To avoid mistakes:} It helps in spotting bad arguments from others (e.g., salespeople, lawyers, evangelists) and prevents being misled. It also aids in making well-thought-out personal decisions.
\end{enumerate}

\subsection{Course Structure}
The course is divided into four parts:
\begin{description}
    \item[Part 1: Analyzing Arguments] \hfill \\
          Distinguishing arguments from non-arguments, identifying argument parts, using standard form, and finding suppressed premises.

    \item[Part 2: Deductive Arguments] \hfill \\
          Focuses on valid, formal structures including propositional logic and categorical logic.

    \item[Part 3: Inductive Arguments] \hfill \\
          Includes statistical generalizations, inference to the best explanation, arguments from analogy, causal reasoning, probability, and decision making.

    \item[Part 4: Fallacies] \hfill \\
          Covers common mistakes like vagueness, ambiguity, irrelevance (ad hominem, appeals to ignorance), and begging the question.
\end{description}

\subsection{Logistics}
\begin{itemize}
    \item \textbf{Communication:} Use discussion forums for questions and interaction. Do not email instructors individually.
\end{itemize}

\subsection{Recommended Reading}
For those who want more detail or are having trouble understanding:
\begin{itemize}
    \item \textbf{Textbook:} \textit{Understanding Arguments}
    \item \textbf{Authors:} Walter Sinnott-Armstrong and Robert Fogelin
\end{itemize}

\noindent\rule{\linewidth}{1pt}

\section{What Is an Argument?}
To understand arguments, we must first distinguish them from what they are not.

\subsection{What Arguments Are Not}
Arguments are distinct from:
\begin{itemize}
    \item \textbf{Physical Fights:} Hitting someone is not arguing.
    \item \textbf{Verbal Fights:} Yelling is not arguing.
    \item \textbf{Abuse:} Name-calling (e.g., ``stupid git'') is not arguing.
    \item \textbf{Complaining:} Expressing emotion is not arguing.
    \item \textbf{Contradiction:} Simply denying a statement is not arguing.
\end{itemize}

\subsection{Defining an Argument}
Monty Python defines an argument as ``a connected series of statements to establish a proposition.'' However, this is too narrow because arguments can also explain propositions we already accept (e.g., mathematical proofs).

\textbf{Broader Definition:} An argument is a connected series of sentences, statements, or propositions, where some are premises and one is the conclusion, and the premises are intended to provide some kind of reason for the conclusion.

This definition highlights:
\begin{enumerate}
    \item \textbf{Parts:} Premises and Conclusion.
    \item \textbf{Medium:} Language.
    \item \textbf{Purpose:} To give a reason for the conclusion.
    \item \textbf{Flexibility:} Covers different kinds of reasons (justification and explanation).
\end{enumerate}

\end{document}