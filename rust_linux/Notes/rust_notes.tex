\documentclass[12pt]{article}
\usepackage[margin=1in]{geometry}
\usepackage{amsmath, amssymb, mathtools}
\usepackage{tikz}
\usetikzlibrary{calc, angles, quotes}
\usetikzlibrary{decorations.pathmorphing}
\usepackage{circuitikz}
\usepackage{lmodern}
\usepackage{hyperref}
\usepackage{caption}
\usepackage{float}
\usepackage{parskip}
\usepackage{tabularx}
\usepackage[utf8]{inputenc}
\usepackage{tgpagella}
\usepackage[T1]{fontenc}
\usepackage{array}
\usepackage{booktabs}
\usepackage{tcolorbox}
\usepackage{enumitem}

\newcommand{\separator}{\noindent\rule{\linewidth}{1pt}}

\title{Rust Fundamentals}
\author{Zubin}
\date{\today}

\begin{document}

\maketitle

\tableofcontents

\newpage

\separator
%-----------------------------------------------------------------


\section{Introduction}

The second-order ODE has the form:
\[ \frac{d^2x}{dt^2} + p(t)\frac{dx}{dt} + q(t)x = g(t) \]
where $g(t)$ is called the inhomogeneous term. Inhomogeneous terms typically represent forces that might be applied to an engineering system. We'll focus on the inhomogeneous second-order ODE with constant coefficients. Inhomogeneous terms we'll consider are exponential functions, sine or cosine functions, and polynomials. We'll also learn about resonance and damped resonance.

Physically, resonance occurs when the frequency of the force matches the natural frequency of the system. Mathematically, resonance happens when the inhomogeneous term is a solution of the homogeneous equation. Applications we'll learn about are the resistor-inductor-capacitor, or RLC, circuit; a mass oscillating on a spring; and the pendulum.




\end{document}
