\documentclass[12pt]{article}
\usepackage[margin=1in]{geometry}
\usepackage{amsmath, amssymb, mathtools}
\usepackage{tikz}
\usepackage{lmodern}
\usepackage{hyperref}
\usepackage{caption}
\usepackage{float}
\usepackage{parskip}
\usepackage{tabularx}
\usepackage[utf8]{inputenc}
\usepackage{tgpagella}
\usepackage[T1]{fontenc}
\usepackage{array}
\usepackage{booktabs}
\usepackage{tcolorbox}
\usepackage{enumitem}

\newcommand{\separator}{\noindent\rule{\linewidth}{1pt}}

\title{History of Linux and the command line}
\author{Zubin}
\date{\today}

\begin{document}

\maketitle

\tableofcontents

\newpage

\separator
%-----------------------------------------------------------------

\section{Compiler vs interpreter}

In the C programming language, we use a compiler. To understand the difference between an interpreter and a compiler, we can use an analogy of landing on a planet where inhabitants speak a strange language called "Gobbledygook". To get a mechanic to repair your spaceship, you need a translator.

\subsection{The Interpreter}
If you choose an interpreter:
\begin{itemize}
    \item \textbf{Process:} The interpreter reads your first instruction, translates it immediately, and the mechanic executes it. Then it reads the second, translates, and executes, and so on.
    \item \textbf{Characteristics:} The interpreter stays with you, translating line by line.
    \item \textbf{Pros:} It allows you to correct mistakes as you go (interactive).
    \item \textbf{Cons:} It is a slow process because the mechanic waits for translation between steps.
    \item \textbf{Etymology:} "Inter" means between. The interpreter is always between your program and the computer.
\end{itemize}

\subsection{The Compiler}
If you choose a compiler:
\begin{itemize}
    \item \textbf{Process:} The compiler takes your complete list of instructions and translates the whole lot at once. It then hands the translated list back to you and leaves.
    \item \textbf{Characteristics:} You hand the complete list to the mechanic, who executes them all in one go very quickly.
    \item \textbf{Pros:} Execution is very fast and efficient.
    \item \textbf{Cons:} Takes extra preparation time initially. If there is a mistake, it is too late to fix it during execution.
    \item \textbf{Etymology:} "Compile" means to pile together. It piles together your entire program and translates it all at once.
\end{itemize}

\subsection{Summary}
\begin{itemize}
    \item \textbf{Interpreter:} Runs slowly, starts right away, allows you to see how things are going.
    \item \textbf{Compiler:} Takes preparation time, but runs very quickly and efficiently.
\end{itemize}

\separator

\end{document}
