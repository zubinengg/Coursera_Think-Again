\documentclass[12pt]{article}

\usepackage[a4paper,margin=1in]{geometry}
\usepackage{amsmath,amssymb}
\usepackage{enumitem}
\usepackage{setspace}

\onehalfspacing

\title{Arguments, Justification, and the Problem of Skeptical Regress}
\author{}
\date{}

\begin{document}

\maketitle

\section{Arguments and Justification}

An \textbf{argument} consists of premises intended to be \textbf{reasons} for a conclusion.
Merely intending premises to support a conclusion is not sufficient.

What matters is \textbf{success}: do the premises actually justify believing the conclusion?

For simplicity, we focus on arguments whose purpose is \textbf{justification}.

\medskip
\textbf{Key Principle:}
\begin{quote}
    An argument can justify belief in its conclusion only if one is justified in believing its premises.
\end{quote}

\section{The Mars Example: Justified Premises}

\textbf{Argument:}
\begin{enumerate}
    \item There is at least one bacterium on Mars.
    \item Therefore, there is life on Mars.
\end{enumerate}

This argument is logically valid: if the premise is true, the conclusion must be true.
However, justification depends on whether the premise itself is justified.

If the premise is merely guessed, then:
\begin{itemize}
    \item the premise is unjustified;
    \item the conclusion is also unjustified.
\end{itemize}

\textbf{Lesson:} A valid argument with unjustified premises does not justify its conclusion.

\section{The Problem of the Skeptical Regress}

If a premise is questioned, it must be supported by another argument.
That argument has its own premises, which also require justification.

This leads to an endless chain of arguments, each requiring further support.

\medskip
\textbf{The Problem of the Skeptical Regress:}
\begin{quote}
    Every premise requires justification by another argument, leading to a regress that seems impossible to terminate.
\end{quote}

\section{Three Responses to the Skeptical Regress}

\subsection{Unjustified Premises}

One response is to begin with premises that are unjustified.

\textbf{Problems:}
\begin{itemize}
    \item Allows guessing.
    \item Anything could be ``proved,'' including falsehoods.
\end{itemize}

\textbf{Conclusion:} This approach fails.

\subsection{Circular Arguments}

Another response is to justify claims in a circular manner.

\textbf{Example:}
\begin{quote}
    There is life on Mars.
    Therefore, there is life on Mars.
\end{quote}

\textbf{Problems:}
\begin{itemize}
    \item Provides no new justification.
    \item Doubting the conclusion forces doubt about the premise.
    \item Circular arguments can support opposing conclusions.
\end{itemize}

\textbf{Conclusion:} Circular justification is inadequate.

\subsection{Infinite Chains of Arguments}

A third response is to allow an infinite chain of justifications.

\textbf{Example:}
\begin{itemize}
    \item At least one bacterium on Mars
    \item At least two bacteria on Mars
    \item At least three bacteria on Mars
    \item and so on, infinitely
\end{itemize}

\textbf{Problem:}
\begin{itemize}
    \item No premise has independent justification.
    \item Infinite length does not produce justification.
\end{itemize}

\textbf{Conclusion:} Infinite regress does not justify belief.

\section{Philosophical Significance}

All three theoretical responses fail:
\begin{itemize}
    \item unjustified premises,
    \item circular arguments,
    \item infinite regress.
\end{itemize}

This raises a serious concern:
\begin{quote}
    How can any argument ever justify belief?
\end{quote}

\section{Everyday Solutions to the Skeptical Regress}

In practice, people resolve the problem pragmatically.

\subsection{Shared Assumptions}

Arguments often begin from assumptions shared by speaker and audience.

\textbf{Example:}
\begin{quote}
    You should buy a Honda because Hondas are reliable.
\end{quote}

This assumes both parties value reliability.

\subsection{Appeal to Authority}

Justification may rely on trusted authorities accepted by the audience.

\textbf{Example:}
\begin{quote}
    Consumer Reports shows that Hondas are reliable.
\end{quote}

\subsection{Handling Objections}

\begin{itemize}
    \item \textbf{Discounting objections}: acknowledging possible error while rejecting its relevance.
    \item \textbf{Guarding claims}: using qualifiers such as ``probably'' or ``likely.''
\end{itemize}

\section{Central Insight}

Justification requires shared starting points.

Arguments succeed only when premises, values, or authorities are accepted by the audience.

More shared assumptions make argumentation easier; fewer shared assumptions make justification difficult or impossible.

\section{Conclusion}

\begin{itemize}
    \item Premises must be justified to justify conclusions.
    \item The skeptical regress shows the limits of pure logical justification.
    \item Philosophical solutions fail in theory.
    \item Everyday reasoning succeeds through shared assumptions, authority, and managing objections.
\end{itemize}

\begin{quote}
    Arguments justify belief only within a shared framework of assumptions.
\end{quote}

\end{document}
