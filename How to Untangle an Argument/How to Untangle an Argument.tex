\documentclass[12pt]{article}
\usepackage[margin=1in]{geometry}
\usepackage{amsmath, amssymb, mathtools}
\usepackage{tikz}
\usepackage{lmodern}
\usepackage{hyperref}
\usepackage{caption}
\usepackage{float}
\usepackage{parskip}
\usepackage{tabularx}
\usepackage[utf8]{inputenc}
\usepackage{tgpagella}
\usepackage[T1]{fontenc}
\usepackage{array}
\usepackage{booktabs}
\usepackage{tcolorbox}
\usepackage{enumitem}

\newcommand{\separator}{\noindent\rule{\linewidth}{1pt}}

\title{Think Again: How to Untangle an Arguments}
\author{Zubin}
\date{\today}

\begin{document}

\maketitle

\tableofcontents

\newpage

\separator
%-----------------------------------------------------------------

\section{Argument Markers}
To spot an argument, we need to understand how language indicates that some sentences are intended as reasons for others. Certain words, which we will call \textbf{argument markers}, signal the presence of an argument by clarifying the relationship between premises and conclusions.

\subsection{Identifying Intent}
The difference between simply stating two facts and presenting an argument lies in the speaker's intent, which is revealed through word choice.
\begin{itemize}
    \item \textbf{Conjunction:} In the sentence, ``I am tall, and I am good at sports,'' the word ``and'' simply conjoins two separate statements. The order can be reversed without changing the meaning.
    \item \textbf{Argument:} In ``I am tall, so I am good at sports,'' the word ``so'' indicates a rational connection. The order matters greatly; reversing it (``I am good at sports, so I am tall'') creates a different argument.
\end{itemize}

\subsection{Types of Argument Markers}
Argument markers fall into two main categories based on what they signal.

\subsubsection{Conclusion Markers}
These words indicate that the statement immediately following them is the conclusion of an argument.
\begin{itemize}
    \item \textbf{Examples:} so, therefore, thus, hence, accordingly.
    \item \textbf{Structure:} [Premise], so [Conclusion].
    \item \textbf{Example:} ``I am tall, \textit{therefore} I am good at sports.''
\end{itemize}

\subsubsection{Reason Markers (Premise Markers)}
These words indicate that the statement immediately following them is a reason (a premise) for a conclusion.
\begin{itemize}
    \item \textbf{Examples:} because, for, as, since, for the reason that.
    \item \textbf{Structure:} [Conclusion] because [Premise].
    \item \textbf{Example:} ``I am good at sports \textit{because} I am tall.''
\end{itemize}

\begin{tcolorbox}[colback=blue!5!white,colframe=blue!75!black,title=Argument Markers]
    \begin{tabularx}{\linewidth}{@{}XX@{}}
        \toprule
        \textbf{Conclusion Markers} & \textbf{Reason (Premise) Markers} \\
        \midrule
        so                          & because                           \\
        therefore                   & for                               \\
        thus                        & as                                \\
        hence                       & since                             \\
        accordingly                 & for the reason that               \\
        \bottomrule
    \end{tabularx}
\end{tcolorbox}

\subsection{The Importance of Context}
A word that functions as an argument marker in one context may not in another. You must analyze the role the word is playing.

\begin{itemize}
    \item \textbf{Example with ``since'':}
          \begin{itemize}
              \item \textit{As a reason marker:} ``I'm good at sports \textbf{since} I am tall.'' (Here, being tall is the reason for being good at sports).
              \item \textit{As a temporal marker:} ``It has been raining \textbf{since} my vacation began.'' (This indicates timing, not a causal or rational link).
          \end{itemize}
    \item \textbf{Example with ``so'':}
          \begin{itemize}
              \item \textit{As a conclusion marker:} ``I am tall, \textbf{so} I am good at sports.''
              \item \textit{As an intensifier:} ``You don't need to eat \textbf{so} much.'' (This does not indicate a conclusion).
          \end{itemize}
\end{itemize}

\subsection{The Substitution Test}
To determine if a word is being used as an argument marker, try substituting it with a clear, unambiguous marker like ``therefore'' (for conclusions) or ``because'' (for reasons).
\begin{itemize}
    \item \textbf{Rule:} If the substitution can be made without changing the fundamental meaning of the sentence, then the original word was likely being used as an argument marker.
    \item \textbf{Example 1:} ``Since he failed out of college, he's unemployed.''
          \begin{itemize}
              \item \textit{Substitute with ``because'':} ``\textbf{Because} he failed out of college, he's unemployed.'' The meaning is preserved.
              \item \textit{Conclusion:} ``Since'' is used as a reason marker here.
          \end{itemize}
    \item \textbf{Example 2:} ``He's so cool!''
          \begin{itemize}
              \item \textit{Substitute with ``therefore'' or ``because'':} ``He's therefore cool'' or ``He's because cool.'' Both are nonsensical and change the meaning.
              \item \textit{Conclusion:} ``So'' is not used as an argument marker here.
          \end{itemize}
\end{itemize}

\subsection{A Special Case: The Word ``If''}
The word ``if,'' often found in ``if...then'' statements (conditionals), is \textbf{not} an argument marker.
\begin{itemize}
    \item An ``if-then'' statement by itself does not make an argument because it does not assert that the ``if'' clause (the antecedent) is true.
    \item \textbf{Conditional Statement (Not an Argument):} ``If I am rich enough, then I can buy a baseball team.'' This doesn't claim I am rich, nor that I can buy a team.
    \item \textbf{Full Argument:} ``If I'm rich enough, I can buy a baseball team. I am rich enough. Therefore, I can buy a baseball team.''
    \item Because the word ``if'' alone does not assert a premise, we do not count it as an argument marker.
\end{itemize}

\separator
\section{Arguments and Argument Markers: Worked Examples}

\subsection*{Question 1}
\textbf{Statement:}
Charles went bald because most men his age go bald.

\textbf{Correct Answer:} Is an argument

\textbf{Reason:}
This sentence claims that the fact that most men his age go bald is a reason that explains why Charles went bald. According to this claim, his age helps us understand why he went bald and makes it less surprising that he went bald.

\subsection*{Question 2}
\textbf{Statement:}
Charles went bald, and most men his age go bald.

\textbf{Correct Answer:} Is not an argument

\textbf{Reason:}
This sentence says only that both facts are true and does not explicitly say that one is a reason for the other.

\subsection*{Question 3}
\textbf{Statement:}
My roommate likes to ski, so I do, too.

\textbf{Correct Answer:} Is an argument

\textbf{Reason:}
This sentence claims that the fact that my roommate likes to ski is a reason that explains why I like to ski. According to this claim, my roommate’s likes help us understand why I like to ski and make it less surprising that I like to ski.

\subsection*{Question 4}
\textbf{Statement:}
My roommate likes to ski, and so do I.

\textbf{Correct Answer:} Is not an argument

\textbf{Reason:}
This sentence says only that both facts are true and does not explicitly say that one is a reason for the other. The word \emph{``so''} here is simply short for \emph{``also''} and does not function as an argument marker.

\subsection*{Question 5}
\textbf{Statement:}
I have been busy since Tuesday.

\textbf{Correct Answer:} Is not an argument

\textbf{Reason:}
The sentence says only that I have been busy after the time or day when it was Tuesday. It does not say that I have been busy \emph{because} that day was Tuesday. The sentence states a single fact and does not contain both a premise and a conclusion. Since an argument requires at least one premise and a conclusion, this sentence is not an argument.

\subsection*{Question 6}
\textbf{Statement:}
I am busy, since my teacher assigned lots of homework.

\textbf{Correct Answer:} Is an argument

\textbf{Reason:}
This sentence claims that the assignment of lots of homework is a reason why I am busy.

\subsection*{Question 7}
\textbf{Statement:}
He apologized, so you should forgive him.

\textbf{Correct Answer:} Conclusion marker

\textbf{Reason:}
The word \emph{``so''} indicates that the sentence following it is a conclusion. The meaning does not change if we replace \emph{``so''} with \emph{``therefore''}:
\begin{quote}
    He apologized. Therefore, you should forgive him.
\end{quote}

\subsection*{Question 8}
\textbf{Statement:}
In view of the fact that he apologized, you should forgive him.

\textbf{Correct Answer:} Premise marker

\textbf{Reason:}
The phrase \emph{``In view of the fact that''} introduces a premise. The meaning does not change if we substitute \emph{``because''} for the phrase:
\begin{quote}
    Because he apologized, you should forgive him.
\end{quote}

\subsection*{Question 9}
\textbf{Statement:}
He apologized. Accordingly, you should forgive him.

\textbf{Correct Answer:} Conclusion marker

\textbf{Reason:}
The word \emph{``accordingly''} signals that what follows is a conclusion. This is shown by the fact that the meaning remains the same if we replace it with \emph{``therefore''}.

\subsection*{Question 10}
\textbf{Statement:}
After he apologizes, you should forgive him.

\textbf{Correct Answer:} Neither

\textbf{Reason:}
The word \emph{``after''} indicates only a temporal relationship, not a logical relationship of premise to conclusion. Replacing it with \emph{``because''} or \emph{``therefore''} changes the meaning of the sentence.

\subsection*{Question 11}
\textbf{Statement:}
Seeing as he apologized, you should forgive him.

\textbf{Correct Answer:} Premise marker

\textbf{Reason:}
The phrase \emph{``seeing as''} introduces a premise. The meaning remains unchanged if we substitute \emph{``because''}:
\begin{quote}
    Because he apologized, you should forgive him.
\end{quote}

\separator

\section{Standard Form}
Now that we've identified arguments, and we've also identified premises and conclusions, we need to put them in order. The actual word order doesn't always tell us the order of the argument. Compare these two sentences: ``Because I am a professor, I teach classes,'' and ``I teach classes, because I am a professor.'' In those two examples, the phrase ``I am a professor'' occurs at the beginning in one instance and at the end in the other. However, they express exactly the same argument: the fact that I'm a professor is a reason why I teach classes.

Contrast both of those with this example: ``I teach classes, so I must be a professor.'' The point there must be something like, ``nobody but professors can teach classes,'' and whether or not that's true, the point here is that this is a different argument from the first one.

We need to represent the difference between these arguments very carefully. To show what is shared by the first two examples that's different from the third, we put the arguments in what's called \textbf{standard form}. It's really easy. Basically, you write the premise, and if there's another premise, you write that down on a new line. Then you draw a line. After the line, you put a ``dot pyramid'' ($\therefore$)—three dots with two at the bottom and one at the top—and then you write the conclusion.

It's also useful to number the premises and the conclusion so that we can refer back to them by number instead of having to repeat them. That's all there is to standard form. You list the premises on different lines, draw a line, add the dot pyramid, and then write the conclusion, numbering all the parts.

This standard form accomplishes what we want: it helps us show what's common to the first two examples that distinguishes them from the third.

The first two examples were, ``I teach classes because I'm a professor,'' and ``Because I'm a professor, I teach classes.'' In standard form, the premise that goes above the line is ``I am a professor,'' and the conclusion that goes below the line is ``I teach classes.''

\begin{tcolorbox}
    \begin{enumerate}[label=(\arabic*)]
        \item I am a professor.
    \end{enumerate}
    \separator
    \begin{enumerate}[label=$\therefore$ (\arabic*), start=2]
        \item I teach classes.
    \end{enumerate}
\end{tcolorbox}

The third example was, ``I teach classes, so I must be a professor.'' Here, ``I am a professor'' is the conclusion that goes below the line, and the premise is ``I teach classes.''

\begin{tcolorbox}
    \begin{enumerate}[label=(\arabic*)]
        \item I teach classes.
    \end{enumerate}
    \separator
    \begin{enumerate}[label=$\therefore$ (\arabic*), start=2]
        \item I must be a professor.
    \end{enumerate}
\end{tcolorbox}

When you put these two arguments in standard form next to each other, it makes the difference absolutely clear. No matter how easy this seems, it's worthwhile to practice with a few exercises to make sure we've got it straight, because this notion of standard form will become important later.

\separator

\section{A Problem for Arguments}
Now that we've learned how to identify an argument and put it in standard form, we also know from the definition of an argument that the premises are intended to be reasons for the conclusion. Intentions are nice, but success is better. What we need to figure out is when a person succeeds in giving premises that really are reasons for the conclusion. For simplicity, let's focus on arguments whose purpose is justification. The question then is: do the premises justify you in believing the conclusion?

\subsection{The Need for Justified Premises}
Imagine that you don't know whether there's any life on Mars. You have no evidence one way or the other. Then you ask a friend, and the friend says, ``Hah, I know there's life on Mars. I can prove it to you. This argument will show you that there's life on Mars: There is at least one bacterium on Mars. Therefore, there is life on Mars.''

Notice that if the premise is true, the conclusion has to be true. And if you and your friend are justified in believing the premise, then you are also justified in believing the conclusion. So, this argument looks pretty good so far. But of course, you have to ask your friend, ``Well, how do you know that there's at least one bacterium on Mars?'' Suppose your friend says, ``Well, I'm just guessing.'' Then, the argument is clearly no good. If there's no reason to believe the premise because your friend is just guessing, then you're not justified in believing that premise. And if you're not justified in believing the premise, then how can that premise make you justified in believing the conclusion?

More generally, an argument cannot justify you in believing the conclusion unless you're justified in accepting the premises of that argument.

\subsection{The Skeptical Regress}
Now, suppose your friend says, ``But I do have a reason for the premise, I do!'' Then, we have to ask what kind of reason it is. At that point, your friend needs to express that reason, and how do we express reasons? In arguments. So your friend needs to give another argument for the premise, where the premise of the first argument is the conclusion of the second argument. But wait a minute, now we've got a problem.

Because that second argument is itself going to have premises, and you have to be justified in believing those. So, the premises of the second argument have to be the conclusion of a third argument, and so on. This problem is called \textbf{The Problem of the Skeptical Regress}, because you regress back from one argument to another, to another, to another. It's hard to see how that regress is ever going to come to an end.

There seem to be only three ways to get around the skeptical regress:
\begin{enumerate}
    \item Start with a premise that's unjustified.
    \item Have a structure where the arguments move in a circle.
    \item The chain of arguments goes back infinitely.
\end{enumerate}

\subsection{Evaluating the Options}

\subsubsection{Unjustified Premises}
The first possibility is to start with a premise that's unjustified. That seems pretty neat, if you can get away with it. But we already saw why that won't work with the Mars example. If you just guess at your premises, you have no reason to believe them. Then, an argument that uses those premises cannot justify you in believing the conclusion.

In addition, you could prove anything if we let you start with unjustified premises. If you can just make up your premises for no reason, then there's no stopping you from believing whatever you want, including things that are obviously false. So it seems to be a real problem to start with premises that are unjustified.

\subsubsection{Circular Arguments}
The second way to respond to the skeptical regress is to use a circular structure. It's kind of neat if you think about it. One claim is justified by another, which is justified by another, which is justified by the first claim. The arguments move in a circle, but that means that every premise has an argument to back it up.

If you think about it a little bit, it'll be obvious that that's no good. This can be shown by looking at the smallest circle there is. Suppose your friend says, ``I can prove there's life on Mars. Here's my argument: There is life on Mars. Therefore, there is life on Mars.''

Clearly that's no good. The reason it's no good is that if you didn't know whether there was life on Mars to begin with, you wouldn't know whether the premise was true, because the premise is the conclusion. So if you're not justified in believing the conclusion to begin with, you're not justified in believing the premise, and that means that the argument didn't really get you anywhere.

In addition, it has the same problem as the first approach: you can use circular arguments to prove anything. You can prove there's life on Mars (``There's life on Mars, therefore there's life on Mars''), and you can prove there's no life on Mars (``There's no life on Mars, therefore there's no life on Mars''). The fact that this structure can be used to prove either conclusion suggests there's a big problem with it.

\subsubsection{Infinite Chains}
The third and final way to get around the skeptical regress is to use an infinite chain of arguments. But if you think about it in a concrete case, you'll see why that's a problem as well. Suppose your friend says, ``There's life on Mars and I can prove it.'' You ask for a reason.
\begin{itemize}
    \item \textbf{Friend:} There's at least one bacterium on Mars, therefore, there's life on Mars.
    \item \textbf{You:} But how do you know there's at least one bacterium?
    \item \textbf{Friend:} There are at least two bacteria on Mars, therefore, there's at least one.
    \item \textbf{You:} But how do you know there are at least two?
    \item \textbf{Friend:} Well, there are at least three, so there are at least two.
    \item \textbf{You:} And how do you know that?
    \item \textbf{Friend:} Well, there are at least four, so there are at least three... and so on, infinitely.
\end{itemize}
An infinite chain of arguments would allow you to ``prove'' that there's life on Mars, even if you have no evidence whatsoever. If the premise you're arguing from doesn't have an independent justification, then the infinite chain is no good at all.

\subsection{Practical Solutions in Everyday Life}
Many people see the skeptical regress as a deep and serious philosophical issue. If unjustified premises, circular arguments, and infinite chains don't work, it's hard to see how any argument could ever justify us in believing anything. Philosophers worry about this a lot.

But how do we solve this problem in everyday life? There are various tricks that people use.

One way is to just start from \textbf{assumptions that everybody shares}. If I say, ``You really ought to buy a Honda, because Hondas are very reliable cars,'' I'm assuming that you and I both agree that reliability is a good quality in a car.

But of course, you might ask, ``Are Hondas reliable?'' Then I might \textbf{appeal to an authority}. ``Well, Consumer Reports has done a study that shows that they're reliable.'' If you accept that authority, the argument works.

Suppose someone raises an objection: ``Consumer Reports has been wrong before.'' Then I need to \textbf{discount that objection}. I might say, ``Well, maybe they have been wrong sometimes, \textit{but} this time their study was very careful.''

I might also just \textbf{guard my claim}. I could say, ``Well, they're \textit{probably} right,'' instead of claiming they are definitely right.

So, we can:
\begin{itemize}
    \item \textbf{Assure} you by citing an authority.
    \item \textbf{Discount} objections.
    \item \textbf{Guard} our premises by weakening them.
\end{itemize}

These are three ways of solving the skeptical regress problem in everyday life that we're going to look at in much more detail in the next three lectures. The main point is that to solve the problem, you have to find some assumptions that you and your audience share. This is tricky and depends on the context. If you're dealing with an audience that shares a lot of your assumptions, arguing is relatively easy. If they don't, it can be impossible.

What these tricks do is give you ways to get the argument going, but they're not going to work in every case. We'll have to look at that as we explore these three different ways to solve the skeptical regress problem in the next three lectures.

\end{document}
