\documentclass[12pt]{article}
\usepackage[margin=1in]{geometry}
\usepackage{amsmath, amssymb, mathtools}
\usepackage{tikz}
\usepackage{lmodern}
\usepackage{hyperref}
\usepackage{caption}
\usepackage{float}
\usepackage{parskip}
\usepackage{tabularx}
\usepackage[utf8]{inputenc}
\usepackage{tgpagella}
\usepackage[T1]{fontenc}
\usepackage{array}
\usepackage{booktabs}
\usepackage{tcolorbox}
\usepackage{enumitem}

\newcommand{\separator}{\noindent\rule{\linewidth}{1pt}}

\title{Think Again: How to Untangle an Arguments}
\author{Zubin}
\date{\today}

\begin{document}

\maketitle

\tableofcontents

\newpage

\separator
%-----------------------------------------------------------------

\section{Argument Markers}
To spot an argument, we need to understand how language indicates that some sentences are intended as reasons for others. Certain words, which we will call \textbf{argument markers}, signal the presence of an argument by clarifying the relationship between premises and conclusions.

\subsection{Identifying Intent}
The difference between simply stating two facts and presenting an argument lies in the speaker's intent, which is revealed through word choice.
\begin{itemize}
    \item \textbf{Conjunction:} In the sentence, ``I am tall, and I am good at sports,'' the word ``and'' simply conjoins two separate statements. The order can be reversed without changing the meaning.
    \item \textbf{Argument:} In ``I am tall, so I am good at sports,'' the word ``so'' indicates a rational connection. The order matters greatly; reversing it (``I am good at sports, so I am tall'') creates a different argument.
\end{itemize}

\subsection{Types of Argument Markers}
Argument markers fall into two main categories based on what they signal.

\subsubsection{Conclusion Markers}
These words indicate that the statement immediately following them is the conclusion of an argument.
\begin{itemize}
    \item \textbf{Examples:} so, therefore, thus, hence, accordingly.
    \item \textbf{Structure:} [Premise], so [Conclusion].
    \item \textbf{Example:} ``I am tall, \textit{therefore} I am good at sports.''
\end{itemize}

\subsubsection{Reason Markers (Premise Markers)}
These words indicate that the statement immediately following them is a reason (a premise) for a conclusion.
\begin{itemize}
    \item \textbf{Examples:} because, for, as, since, for the reason that.
    \item \textbf{Structure:} [Conclusion] because [Premise].
    \item \textbf{Example:} ``I am good at sports \textit{because} I am tall.''
\end{itemize}

\begin{tcolorbox}[colback=blue!5!white,colframe=blue!75!black,title=Argument Markers]
    \begin{tabularx}{\linewidth}{@{}XX@{}}
        \toprule
        \textbf{Conclusion Markers} & \textbf{Reason (Premise) Markers} \\
        \midrule
        so                          & because                           \\
        therefore                   & for                               \\
        thus                        & as                                \\
        hence                       & since                             \\
        accordingly                 & for the reason that               \\
        \bottomrule
    \end{tabularx}
\end{tcolorbox}

\subsection{The Importance of Context}
A word that functions as an argument marker in one context may not in another. You must analyze the role the word is playing.

\begin{itemize}
    \item \textbf{Example with ``since'':}
          \begin{itemize}
              \item \textit{As a reason marker:} ``I'm good at sports \textbf{since} I am tall.'' (Here, being tall is the reason for being good at sports).
              \item \textit{As a temporal marker:} ``It has been raining \textbf{since} my vacation began.'' (This indicates timing, not a causal or rational link).
          \end{itemize}
    \item \textbf{Example with ``so'':}
          \begin{itemize}
              \item \textit{As a conclusion marker:} ``I am tall, \textbf{so} I am good at sports.''
              \item \textit{As an intensifier:} ``You don't need to eat \textbf{so} much.'' (This does not indicate a conclusion).
          \end{itemize}
\end{itemize}

\subsection{The Substitution Test}
To determine if a word is being used as an argument marker, try substituting it with a clear, unambiguous marker like ``therefore'' (for conclusions) or ``because'' (for reasons).
\begin{itemize}
    \item \textbf{Rule:} If the substitution can be made without changing the fundamental meaning of the sentence, then the original word was likely being used as an argument marker.
    \item \textbf{Example 1:} ``Since he failed out of college, he's unemployed.''
          \begin{itemize}
              \item \textit{Substitute with ``because'':} ``\textbf{Because} he failed out of college, he's unemployed.'' The meaning is preserved.
              \item \textit{Conclusion:} ``Since'' is used as a reason marker here.
          \end{itemize}
    \item \textbf{Example 2:} ``He's so cool!''
          \begin{itemize}
              \item \textit{Substitute with ``therefore'' or ``because'':} ``He's therefore cool'' or ``He's because cool.'' Both are nonsensical and change the meaning.
              \item \textit{Conclusion:} ``So'' is not used as an argument marker here.
          \end{itemize}
\end{itemize}

\subsection{A Special Case: The Word ``If''}
The word ``if,'' often found in ``if...then'' statements (conditionals), is \textbf{not} an argument marker.
\begin{itemize}
    \item An ``if-then'' statement by itself does not make an argument because it does not assert that the ``if'' clause (the antecedent) is true.
    \item \textbf{Conditional Statement (Not an Argument):} ``If I am rich enough, then I can buy a baseball team.'' This doesn't claim I am rich, nor that I can buy a team.
    \item \textbf{Full Argument:} ``If I'm rich enough, I can buy a baseball team. I am rich enough. Therefore, I can buy a baseball team.''
    \item Because the word ``if'' alone does not assert a premise, we do not count it as an argument marker.
\end{itemize}

\separator
\section{Arguments and Argument Markers: Worked Examples}

\subsection*{Question 1}
\textbf{Statement:}
Charles went bald because most men his age go bald.

\textbf{Correct Answer:} Is an argument

\textbf{Reason:}
This sentence claims that the fact that most men his age go bald is a reason that explains why Charles went bald. According to this claim, his age helps us understand why he went bald and makes it less surprising that he went bald.

\subsection*{Question 2}
\textbf{Statement:}
Charles went bald, and most men his age go bald.

\textbf{Correct Answer:} Is not an argument

\textbf{Reason:}
This sentence says only that both facts are true and does not explicitly say that one is a reason for the other.

\subsection*{Question 3}
\textbf{Statement:}
My roommate likes to ski, so I do, too.

\textbf{Correct Answer:} Is an argument

\textbf{Reason:}
This sentence claims that the fact that my roommate likes to ski is a reason that explains why I like to ski. According to this claim, my roommate’s likes help us understand why I like to ski and make it less surprising that I like to ski.

\subsection*{Question 4}
\textbf{Statement:}
My roommate likes to ski, and so do I.

\textbf{Correct Answer:} Is not an argument

\textbf{Reason:}
This sentence says only that both facts are true and does not explicitly say that one is a reason for the other. The word \emph{``so''} here is simply short for \emph{``also''} and does not function as an argument marker.

\subsection*{Question 5}
\textbf{Statement:}
I have been busy since Tuesday.

\textbf{Correct Answer:} Is not an argument

\textbf{Reason:}
The sentence says only that I have been busy after the time or day when it was Tuesday. It does not say that I have been busy \emph{because} that day was Tuesday. The sentence states a single fact and does not contain both a premise and a conclusion. Since an argument requires at least one premise and a conclusion, this sentence is not an argument.

\subsection*{Question 6}
\textbf{Statement:}
I am busy, since my teacher assigned lots of homework.

\textbf{Correct Answer:} Is an argument

\textbf{Reason:}
This sentence claims that the assignment of lots of homework is a reason why I am busy.

\subsection*{Question 7}
\textbf{Statement:}
He apologized, so you should forgive him.

\textbf{Correct Answer:} Conclusion marker

\textbf{Reason:}
The word \emph{``so''} indicates that the sentence following it is a conclusion. The meaning does not change if we replace \emph{``so''} with \emph{``therefore''}:
\begin{quote}
    He apologized. Therefore, you should forgive him.
\end{quote}

\subsection*{Question 8}
\textbf{Statement:}
In view of the fact that he apologized, you should forgive him.

\textbf{Correct Answer:} Premise marker

\textbf{Reason:}
The phrase \emph{``In view of the fact that''} introduces a premise. The meaning does not change if we substitute \emph{``because''} for the phrase:
\begin{quote}
    Because he apologized, you should forgive him.
\end{quote}

\subsection*{Question 9}
\textbf{Statement:}
He apologized. Accordingly, you should forgive him.

\textbf{Correct Answer:} Conclusion marker

\textbf{Reason:}
The word \emph{``accordingly''} signals that what follows is a conclusion. This is shown by the fact that the meaning remains the same if we replace it with \emph{``therefore''}.

\subsection*{Question 10}
\textbf{Statement:}
After he apologizes, you should forgive him.

\textbf{Correct Answer:} Neither

\textbf{Reason:}
The word \emph{``after''} indicates only a temporal relationship, not a logical relationship of premise to conclusion. Replacing it with \emph{``because''} or \emph{``therefore''} changes the meaning of the sentence.

\subsection*{Question 11}
\textbf{Statement:}
Seeing as he apologized, you should forgive him.

\textbf{Correct Answer:} Premise marker

\textbf{Reason:}
The phrase \emph{``seeing as''} introduces a premise. The meaning remains unchanged if we substitute \emph{``because''}:
\begin{quote}
    Because he apologized, you should forgive him.
\end{quote}
\end{document}
